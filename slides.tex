\documentclass[usenames, dvipsnames]{beamer}

\beamertemplatenavigationsymbolsempty

\usepackage[l2tabu, orthodox]{nag}

\usepackage[utf8]{inputenc}
\usepackage[T1]{fontenc}

\usepackage[ngerman]{babel}

\usepackage{amsmath}
\usepackage{amssymb}
\usepackage{amsthm}
\usepackage{mathtools}
\usepackage{physics}
\usepackage{centernot}

\usepackage{csquotes}
\usepackage{lmodern}
\usepackage{microtype}
\usepackage{stmaryrd}

\usepackage{parskip}
\usepackage[export]{adjustbox}
\usepackage{framed}

\usepackage{graphicx}
\usepackage{faktor}

\usepackage{tikz-cd}

\newtheorem{aufgabe}{Aufgabe}
\newtheorem{proposition}{Proposition}
\newtheorem{hilfssatz}{Hilfssatz}
\newtheorem{satz}{Satz}
\newtheorem{defle}{Definition und Hilfssatz}

\MakeOuterQuote{"}

\newcommand{\defiff}{\mathrel{\vcentcolon\Longleftrightarrow}}
\newcommand{\compl}[1]{#1^\mathsf{c}}
\newcommand{\cl}[1]{\mathsf{#1}}
\newcommand{\co}[1]{\text{co-}\cl{#1}}
\newcommand{\np}{\cl{NP}}
\DeclareMathOperator{\opt}{OPT}
\DeclareMathOperator{\alg}{\mathcal{A}}

\newcommand\ccat\mathsf
\newcommand\cat\mathcal
\newcommand{\down}[1]{{\downarrow}#1}
\newcommand{\p}[1]{\iftoggle{proofs}{#1}{}}
\newcommand{\n}{\pgfmatrixnextcell}
\newcommand\nat\Rightarrow

\DeclareMathOperator{\colim}{colim}
\DeclareMathOperator{\id}{id}

\DeclarePairedDelimiterX\set[1]\lbrace\rbrace{\def\given{\;\delimsize\vert\;}#1}
\DeclarePairedDelimiter\godel{\langle}{\rangle}
\title{Der Einbettungssatz}
\subtitle{Seminar Kategorientheorie}
\author{Markus Himmel}
\date{19. Juli 2018}

\begin{document}
	\begin{frame}
		\maketitle
	\end{frame}
	\begin{frame}{Das Ziel\ldots}
		\begin{satz}[von der volltreuen Einbettung]
			Für jede kleine abelsche Kategorie existieren ein Ring $R$ und eine volltreue
			und exakte Einbettung in die Kategorie $\ccat{Mod}_R$ der (links-)$R$-Moduln.
		\end{satz}
	\end{frame}
	\begin{frame}{Schlachtplan}
		\begin{hilfssatz}
			Es sei $F\colon\cat{D}\to\ccat{Ab}^{\cat{A}}$ ein filtriertes Diagramm
			linksexakter additiver Funktoren. Dann ist auch $\colim_{D\in\cat{D}} F(D)$
			linksexakt und additiv.
		\end{hilfssatz}\pause
		\begin{definition}[Die Einbettung in $\ccat{Ab}$]
			Wir definieren die Einbettung $U\colon\cat{A}\to\ccat{Ab}$ durch
			$U\coloneqq \colim_{D\in\cat{D}}\cat{A}(\phi(D), {-})$.
			Die Kategorie $\cat{D}$ und der Funktor $\phi\colon\cat{D}\to\cat{A}$ sind
			noch zu konstruieren.
		\end{definition}
	\end{frame}
	\begin{frame}{Wunschliste an $\cat{D}$ und $\phi$}
		\begin{hilfssatz}
			Ist $\cat{D}$ kofiltriert, so ist $U$ linksexakt.
		\end{hilfssatz}\pause
		\begin{hilfssatz}
			Es sei $\cat{D}$ kofiltriert. Weiter besitze jeder Epimorphismus
			$f\colon A\twoheadrightarrow\phi(D)$ in $\cat{A}$ die Darstellung $f = \phi(d)$, wobei
			$d\colon D'\to D$ in $\cat{D}$. Dann erhält $U$ Epimorphismen.
		\end{hilfssatz}\pause
		\begin{hilfssatz}
			Es sei $\cat{D}$ kofiltriert. Weiter gelte:
			\begin{enumerate}
				\item $\forall A\in\cat{A}\ \exists D\in\cat{D}: A=\phi(D)$;
				\item $\phi(d)$ ist ein Epimorphismus in $\cat{A}$ für jeden Morphismus $d$ in $\cat{D}$.
			\end{enumerate}
			Dann ist $U$ treu.
		\end{hilfssatz}
	\end{frame}
	\begin{frame}{Die halbe Miete}
		\begin{satz}[von der treuen Einbettung]\label{abeb}
			Jede kleine abelsche Kategorie lässt sich treu und exakt in die Kategorie
			der abelschen Gruppen einbetten.
		\end{satz}
	\end{frame}
	\begin{frame}{Fun Facts (1)}
		\begin{hilfssatz}
			Sind $D_1, D_2\in\cat{D}$ Objekte mit $D_1\leq D_2$, so ist die kanonische Faktorisierung
			\[
				\begin{tikzcd}
					\phi(D_1) \arrow[r] \n \lim_{D_1<D\leq D_2}\phi(D)
				\end{tikzcd}
			\]
			ein Epimorphismus.
		\end{hilfssatz}
	\end{frame}
	\begin{frame}{Fun Facts (2)}
		\begin{hilfssatz}
			Es sei $D_0\in\cat{D}$ ein Objekt und $\Gamma\colon \down{D_0}\to\cat{A}$
			ein Funktor, wobei für alle $D_1\leq D_2\leq D_0$ die Faktorisierung
			\[
				\begin{tikzcd}
					\Gamma(D_1) \arrow[r] \n \lim_{D_1<D\leq D_2} \Gamma(D)
				\end{tikzcd}
			\]
			ein Epimorphismus sei. Weiter sei $F\colon\cat{A}\to\ccat{Ab}$ ein exakter
			Funktor und $\alpha_{D_0}\colon \cat{A}(\Gamma D_0, {-})\nat F$
			eine natürliche Transformation. Dann besitzt $\alpha_{D_0}$ die
			Faktorisierung
			\[
				\begin{tikzcd}
					\cat{A}(\Gamma D_0, {-}) \arrow[Rightarrow]{r}{s_{D_0}} \n \colim_{D\leq D_0}\cat{A}(\Gamma D, {-}) \arrow[Rightarrow]{r}{\alpha} \n F.
				\end{tikzcd}
			\]
			$s_{D_0}$ ist hierbei der kanonische Morphismus des Colimes.
		\end{hilfssatz}
	\end{frame}
	\begin{frame}{Der letzte Schrei}
		\begin{satz}[von der volltreuen Einbettung]
			Für jede kleine abelsche Kategorie existieren ein Ring $R$ und eine volltreue
			und exakte Einbettung in die Kategorie $\ccat{Mod}_R$ der (links-)$R$-Moduln.
		\end{satz}
	\end{frame}
\end{document}
