\documentclass[a4paper, parskip=half]{scrartcl}

\usepackage[l2tabu, orthodox]{nag}

\usepackage[utf8]{inputenc}
\usepackage[T1]{fontenc}

\usepackage[ngerman]{babel}

\usepackage{amsmath}
\usepackage{amssymb}
\usepackage[thmmarks, amsmath]{ntheorem}
\usepackage{mathtools}
\usepackage{physics}
\usepackage{centernot}

\usepackage{csquotes}
\usepackage{lmodern}
\usepackage{microtype}
\usepackage{enumitem}
\usepackage{stmaryrd}
\usepackage{tikz-cd}
\usepackage{faktor}
\usepackage{etoolbox}

\usepackage[hidelinks]{hyperref}

\theoremstyle{marginbreak}
\theorembodyfont{\normalfont}
\newtheorem{remark}{Bemerkung}[section]
\newtheorem{proposition}[remark]{Proposition}
\newtheorem{theorem}[remark]{Satz}
\newtheorem{lemma}[remark]{Hilfssatz}
\newtheorem{definition}[remark]{Definition}
\newtheorem{defre}[remark]{Definition und Bemerkung}
\newtheorem{defle}[remark]{Definition und Hilfssatz}
\newtheorem{re}[remark]{Erinnerung}
\newtheorem{cor}[remark]{Folgerung}
\newtheorem{corde}[remark]{Folgerung und Definition}
\newtheorem{example}[remark]{Beispiel}

\theoremstyle{nonumberplain}
\theoremheaderfont{\normalfont\itshape}
\theorembodyfont{\footnotesize}
\theoremsymbol{\ensuremath{\Box}}
\newtheorem{proof}{Beweis.}


\MakeOuterQuote{"}

\newtoggle{proofs}
\toggletrue{proofs}

\newcommand{\defiff}{\mathrel{\vcentcolon\Longleftrightarrow}}
\newcommand\ccat\mathsf
\newcommand\cat\mathcal
\newcommand{\down}[1]{{\downarrow}#1}
\newcommand{\p}[1]{\iftoggle{proofs}{#1}{}}
\newcommand{\n}{\pgfmatrixnextcell}

\DeclareMathOperator{\colim}{colim}
\DeclareMathOperator{\id}{id}

\DeclarePairedDelimiterX\set[1]\lbrace\rbrace{\def\given{\;\delimsize\vert\;}#1}

\subtitle{Seminar Kategorientheorie}
\title{Der Einbettungssatz}
\author{Markus Himmel}
\date{19. Juli 2018}

\begin{document}
	\maketitle
	\setcounter{section}{-1}

	\section{Vorbereitungen}
		\begin{proposition}[Explizite Berechnung kleiner filtrierter Colimiten in $\ccat{Set}$]
			Es sei $\cat{C}$ klein und filtriert und $F\colon \cat{C}\to\ccat{Set}$ ein
			Funktor. Es ist $\colim_\cat{C} F = L$, wobei
			\[
				L=\faktor{\coprod_{C\in\cat{C}}FC}{\approx},\quad s_C\colon FC\to L;\ x\mapsto [x]_\approx.
			\]
			Die Äquivalenzrelation $\approx$ ist gegeben durch
			\[
				(x\in FC) \approx (x'\in FC') \defiff \exists C''\in\cat{C}, f\colon C\to C'', g\colon C'\to C'': Ff(x) = Fg(x').
			\]
		\end{proposition}
		\p{\begin{proof}
			Zu zeigen sind:
			\begin{enumerate}
				\item $\approx$ ist eine Äquivalenzrelation.
				%\item $L$ ist eine abelsche Gruppe.
				\item Der gegebene Kegel ist ein Grenzkegel.
			\end{enumerate}
			\begin{enumerate}[label=Zu \arabic*:]
					\begin{minipage}{0.7\textwidth}
					\item Klar: $\approx$ ist reflexiv und symmetrisch. Sei $(x\in FC_1)
					\approx (x' \in FC_2)$, $(x'\in FC_2) \approx (x''\in FC_3)$.
					Es existieren also Morphismen $f, g, h, k$, sodass
					$Ff(x) = Fg(x')$, $Fh(x')=Fk(x'')$. Da $\cat{C}$ filtriert,
					existiert ein Cokegel $(\alpha_i\colon C_i\to C_6)$ über dem
					gegebenen Diagramm. Insbesondere haben wir also
					\begin{align*}
						F\alpha_4 \circ Ff(x) &= F\alpha_4\circ Fg(x') = F\alpha_2(x')\\
							&= F\alpha_5\circ Fh(x') = F\alpha_5\circ Fk(x'').
					\end{align*}
					Dies zeigt $x \approx x''$.
					\end{minipage}
					\begin{minipage}{0.25\textwidth}
						\[\begin{tikzcd}[sep=1em]
							C_1 \ar{dr}{f} \n \n\\
							\n C_4 \ar{dr}{\alpha_4} \n\\
							C_2 \ar{dr}{h} \ar{ur}{g} \ar{rr}{\alpha_2} \n \n C_6\\
							\n C_5 \ar{ur}{\alpha_5} \n\\
							C_3 \ar{ur}{k} \n \n
						\end{tikzcd}\]
					\end{minipage}
				%\item Es sei $(x\in FC_1)\approx (x'\in FC_2)$, $(y\in FC_3)\approx(y'\in FC_4)$.
				\item Es seien $C, C'\in\cat{C}$, $f\colon C\to C'$, $x\in C$.
					Wegen $Ff(x) = F(\id_{C'})(Ff(x))$ gilt $s_C(x) = [x]_\approx
					= [Ff(x)]_\approx = s_{C'}(Ff(x)) \implies s_C = s_{C'}\circ Ff$.
					Also bilden die $s_C$ einen Kegel. Es sei ein weiterer Kegel
					gegeben durch $(t_C\colon FC\to M)_{C\in\cat{C}}$. Definiere
					$t\colon L\to M$ durch $t([x]_\approx)\coloneqq t_C(x)$ für
					$x\in FC$. Dies ist wohldefiniert, denn für $x'\in FC'$ mit
					$[x']_\approx = [x]_\approx$ existieren $f\colon C\to C''$,
					$g\colon C'\to C''$ mit $Ff(x) = Fg(x')$. Da $t$ ein Kegel ist, folgt
					\[
						t_C(x) = t_{C''}\circ Ff(x) = t_{C''}\circ Fg(x') = t_{C'}(x').
					\]
					$t\circ s_C=t_C$ gilt per Konstruktion und $t$ wird durch diese
					Identität eindeutig festgelegt.
			\end{enumerate}
		\end{proof}}
		\begin{proposition}
			Es sei $\cat{C}$ eine Kategorie mit Pullbacks.
			\[
				\begin{tikzcd}[sep=4em]
					A \ar{r}{a} \ar[phantom]{dr}{\text{(I)}} \ar{d}{c} & B \ar[phantom]{dr}{\text{(II)}} \ar{r}{b} \ar{d}{d} & C \ar{d}{e}\\
					D \ar{r}{f} & E \ar{r}{g} & F
				\end{tikzcd}
			\]
			\begin{enumerate}
				\item Falls (I) und (II) Pullbacks sind, so ist auch das äußere
					Rechteck ein Pullback.
				\item Falls (II) und das äußere Quadrat Pullbacks sind, so ist auch
					(I) ein Pullback.
			\end{enumerate}
		\end{proposition}
		\p{\begin{proof}
			\begin{enumerate}
				\item Es seien $x\colon X\to D$, $y\colon X\to C$ mit $g\circ f\circ x = e\circ y$.
					Pullback (II) liefert ein eindeutiges $z\colon X\to B$, sodass
					$b\circ z = y$, $d\circ z = f\circ x$. Letztere Gleichheit und Pullback
					(I) gibt ein eindeutiges $w\colon X\to A$, sodass $a\circ w = z$, $c\circ w = x$.
					Insbesondere gilt $b\circ a\circ w = b\circ z = y$. Es sei $w'\colon X\to A$ ein
					weiterer Morphismus mit $b\circ a\circ w' = y$ und $c\circ w' = x$. Es gilt also
					$b\circ (a\circ w') = y = b\circ (a\circ w)$ und
					$d\circ(a\circ w') = f\circ (c\circ w') = f\circ x = f\circ c\circ w = d\circ (a\circ w)$.
					Pullback (II) liefert $a\circ w'=a\circ w$. Weiter ist $c\circ w'= x = c\circ w$.
					Pullback (I) liefert $w=w'$.
				\item Es sei $(A', c', a')$ der Pullback von $f$ und $d$. Mit (1) und da (II) ein Pullback ist, ist $(A', c', b\circ a')$ ein Pullback von
					$(g\circ f, e)$. Es ist jedoch auch $(A, c, b\circ a)$ ein Pullback von $(g\circ f, e)$.
					Da Limiten eindeutig sind, folgt $A\cong A'$.
			\end{enumerate}
		\end{proof}}
		\begin{proposition}
			Falls in einer Kategorie $\cat{C}$ der Pullback eines Coequalizers mit sich
			selbst existiert, so ist dieser Coequalizer gerade der Coequalizer dieses Pullbacks.
		\end{proposition}

	\section{Der Einbettungssatz}
		Es sei stets $\cat{A}$ eine kleine abelsche Kategorie.
		\begin{proposition}
			Es sei $F\colon\cat{D}\to\ccat{Ab}^{\cat{A}}$ ein filtriertes Diagramm
			linksexakter additiver Funktoren. Dann ist auch $\colim_{D\in\cat{D}} F(D)$
			linksexakt und additiv.
		\end{proposition}
		\begin{definition}[Die Einbettung]
			Für eine Kategorie $\cat{D}$ und einen Funktor $\phi\colon\cat{D}\to\cat{A}$
			definiere $U\coloneqq\colim_{D\in\cat{D}} (Y\circ\phi)(D)$, wobei $Y$ die
			kontravariante Yoneda-Einbettung bezeichnet.
		\end{definition}
		\begin{lemma}
			Ist $\cat{D}$ kofiltriert, so ist $U$ linksexakt.
		\end{lemma}
		\begin{lemma}
			Es sei $\cat{D}$ kofiltriert. Weiter besitze jeder Epimorphismus
			$f\colon A\to\phi(D)$ in $\cat{A}$ die Darstellung $f = \phi(d)$, wobei
			$d\colon D'\to D$ in $\cat{D}$. Dann erhält $U$ Epimorphismen.
		\end{lemma}
		\begin{lemma}
			Es sei $\cat{D}$ kofiltriert. Weiter gelte:
			\begin{enumerate}[label=(\arabic*)]
				\item $\forall A\in\cat{A}\ \exists D\in\cat{D}: A=\phi(D)$;
				\item $\phi(d)$ ist ein Epimorphismus in $\cat{A}$ für jeden Morphismus $d$ in $\cat{D}$.
			\end{enumerate}
			Dann ist $U$ treu.
		\end{lemma}
		\begin{theorem}[von der treuen Einbettung]\label{abeb}
			Jede kleine abelsche Kategorie lässt sich treu und exakt in die Kategorie
			der abelschen Gruppen einbetten.
		\end{theorem}

		Die Aussagen von \ref{fin}, \ref{fact} und \ref{func} beziehen sich auf
		die im Beweis von \ref{abeb} konstruierten
		Wahlen für $\cat{D}$ und $\phi$.
		\begin{defle}\label{fin}
			Für alle Objekte $D_1,D_2\in\cat{D}$ ist das Segment
				\[
					[D_1, D_2] \coloneqq\set{D\in\cat{D}\given D_1\leq D\leq D_2}
				\]
			endlich.
		\end{defle}
		\begin{lemma}\label{fact}
			Sind $D_1, D_2\in\cat{D}$ Objekte, so ist die kanonische Faktorisierung
			\[
				\begin{tikzcd}
					\phi(D_1) \arrow[r] & \lim_{D_1<D\leq D_2}\phi(D)
				\end{tikzcd}
			\]
			ein Epimorphismus.
		\end{lemma}
		\begin{lemma}\label{func}
			Es sei $D_0\in\cat{D}$ ein Objekt und $\Gamma\colon \down{D_0}\to\cat{A}$
			ein Funktor, wobei für alle $D_1\leq D_2\leq D_0$ die Faktorisierung
			\[
				\begin{tikzcd}
					\Gamma(D_1) \arrow[r] & \lim_{D_1<D\leq D_2} \Gamma(D)
				\end{tikzcd}
			\]
			ein Epimorphismus sei. Weiter sei $F\colon\cat{A}\to\ccat{Ab}$ ein
			Funktor und $\alpha_{D_0}\colon \cat{A}(\Gamma D_0, {-})\Rightarrow F$
			eine natürliche Transformation. Dann besitzt $\alpha_{D_0}$ die
			Faktorisierung
			\[
				\begin{tikzcd}
					\cat{A}(\Gamma D_0, {-}) \arrow{r}{s_{D_0}} & \colim_{D\leq D_0}\cat{A}(\Gamma D, {-}) \arrow{r}{\alpha} & F.
				\end{tikzcd}
			\]
			$s_{D_0}$ ist hierbei der kanonische Morphismus des Colimes.
		\end{lemma}
		\begin{theorem}[von der volltreuen Einbettung; Einbettungssatz von Freyd-Mitchell]
			Für jede kleine abelsche Kategorie existieren ein Ring $R$ und eine volltreue
			und exakte Einbettung in die Kategorie $\ccat{Mod}_R$ der $R$-Moduln.
		\end{theorem}
\end{document}
