\documentclass[a4paper, parskip=half]{scrartcl}

\usepackage[l2tabu, orthodox]{nag}

\usepackage[utf8]{inputenc}
\usepackage[T1]{fontenc}

\usepackage[ngerman]{babel}

\usepackage{amsmath}
\usepackage{amssymb}
\usepackage{ntheorem}
\usepackage{mathtools}
\usepackage{physics}
\usepackage{centernot}

\usepackage{csquotes}
\usepackage{lmodern}
\usepackage{microtype}
\usepackage{enumitem}
\usepackage{stmaryrd}
\usepackage{tikz-cd}
\usepackage{faktor}

\usepackage[hidelinks]{hyperref}

\theoremstyle{marginbreak}
\theorembodyfont{\normalfont}
\newtheorem{remark}{Bemerkung}[section]
\newtheorem{proposition}[remark]{Proposition}
\newtheorem{theorem}[remark]{Satz}
\newtheorem{lemma}[remark]{Hilfssatz}
\newtheorem{definition}[remark]{Definition}
\newtheorem{defre}[remark]{Definition und Bemerkung}
\newtheorem{defle}[remark]{Definition und Hilfssatz}
\newtheorem{re}[remark]{Erinnerung}
\newtheorem{cor}[remark]{Folgerung}
\newtheorem{corde}[remark]{Folgerung und Definition}
\newtheorem{example}[remark]{Beispiel}

\MakeOuterQuote{"}

\newcommand{\defiff}{\mathrel{\vcentcolon\Longleftrightarrow}}
\newcommand\ccat\mathsf
\newcommand\cat\mathcal
\newcommand{\down}[1]{{\downarrow}#1}

\DeclareMathOperator{\colim}{colim}

\DeclarePairedDelimiterX\set[1]\lbrace\rbrace{\def\given{\;\delimsize\vert\;}#1}

\subtitle{Seminar Kategorientheorie}
\title{Der Einbettungssatz}
\author{Markus Himmel}
\date{19. Juli 2018}

\begin{document}
	\maketitle
	\setcounter{section}{-1}

	\section{Vorbereitungen}
		\begin{proposition}[Explizite Berechnung kleiner filtrierter Colimiten in $\ccat{Ab}$]
			Es sei $\cat{C}$ klein und filtriert und $F\colon \cat{C}\to\ccat{Ab}$ ein
			Funktor. Es ist $\colim_\cat{C} F = L$, wobei
			\[
				L=\faktor{\coprod_{C\in\cat{C}}FC}{\approx},\quad s_C\colon FC\to L;\ x\mapsto [x]_\approx.
			\]
			Die Äquivalenzrelation $\approx$ ist gegeben durch
			\[
				(x\in FC) \approx (x'\in FC') \defiff \exists C''\in\cat{C}, f\colon C\to C'', g\colon C'\to C'': Ff(x) = Fg(x').
			\]
		\end{proposition}
		\begin{proposition}
			Es sei $\cat{C}$ eine Kategorie mit Pullbacks.
			\[
				\begin{tikzcd}[sep=4em]
					A \ar{r}{a} \ar[phantom]{dr}{\text{(I)}} \ar{d}{c} & B \ar[phantom]{dr}{\text{(II)}} \ar{r}{b} \ar{d}{d} & C \ar{d}{e}\\
					D \ar{r}{f} & E \ar{r}{g} & F
				\end{tikzcd}
			\]
			Falls sowohl das äußere Rechteck als auch das Quadrat (II) ein Pullback sind, so ist
			auch das Quadrat (I) ein Pullback.
		\end{proposition}
		\begin{proposition}
			Falls in einer Kategorie $\cat{C}$ der Pullback eines Coequalizers mit sich
			selbst existiert, so ist dieser Coequalizer gerade der Coequalizer dieses Pullbacks.
		\end{proposition}

	\section{Der Einbettungssatz}
		Es sei stets $\cat{A}$ eine kleine abelsche Kategorie.
		\begin{proposition}
			Es sei $F\colon\cat{D}\to\ccat{Ab}^{\cat{A}}$ ein filtriertes Diagramm
			linksexakter additiver Funktoren. Dann ist auch $\colim_{D\in\cat{D}} F(D)$
			linksexakt und additiv.
		\end{proposition}
		\begin{definition}[Die Einbettung]
			Für eine Kategorie $\cat{D}$ und einen Funktor $\phi\colon\cat{D}\to\cat{A}$
			definiere $U\coloneqq\colim_{D\in\cat{D}} (Y\circ\phi)(D)$, wobei $Y$ die
			kontravariante Yoneda-Einbettung bezeichnet.
		\end{definition}
		\begin{lemma}
			Ist $\cat{D}$ kofiltriert, so ist $U$ linksexakt.
		\end{lemma}
		\begin{lemma}
			Es sei $\cat{D}$ kofiltriert. Weiter besitze jeder Epimorphismus
			$f\colon A\to\phi(D)$ in $\cat{A}$ die Darstellung $f = \phi(d)$, wobei
			$d\colon D'\to D$ in $\cat{D}$. Dann erhält $U$ Epimorphismen.
		\end{lemma}
		\begin{lemma}
			Es sei $\cat{D}$ kofiltriert. Weiter gelte:
			\begin{enumerate}[label=(\arabic*)]
				\item $\forall A\in\cat{A}\ \exists D\in\cat{D}: A=\phi(D)$;
				\item $\phi(d)$ ist ein Epimorphismus in $\cat{A}$ für jeden Morphismus $d$ in $\cat{D}$.
			\end{enumerate}
			Dann ist $U$ treu.
		\end{lemma}
		\begin{theorem}[von der treuen Einbettung]\label{abeb}
			Jede kleine abelsche Kategorie lässt sich treu und exakt in die Kategorie
			der abelschen Gruppen einbetten.
		\end{theorem}

		Die Aussagen von \ref{fin}, \ref{fact} und \ref{func} beziehen sich auf
		die im Beweis von \ref{abeb} konstruierten
		Wahlen für $\cat{D}$ und $\phi$.
		\begin{defle}\label{fin}
			Für alle Objekte $D_1,D_2\in\cat{D}$ ist das Segment
				\[
					[D_1, D_2] \coloneqq\set{D\in\cat{D}\given D_1\leq D\leq D_2}
				\]
			endlich.
		\end{defle}
		\begin{lemma}\label{fact}
			Sind $D_1, D_2\in\cat{D}$ Objekte, so ist die kanonische Faktorisierung
			\[
				\begin{tikzcd}
					\phi(D_1) \arrow[r] & \lim_{D_1<D\leq D_2}\phi(D)
				\end{tikzcd}
			\]
			ein Epimorphismus.
		\end{lemma}
		\begin{lemma}\label{func}
			Es sei $D_0\in\cat{D}$ ein Objekt und $\Gamma\colon \down{D_0}\to\cat{A}$
			ein Funktor, wobei für alle $D_1\leq D_2\leq D_0$ die Faktorisierung
			\[
				\begin{tikzcd}
					\Gamma(D_1) \arrow[r] & \lim_{D_1<D\leq D_2} \Gamma(D)
				\end{tikzcd}
			\]
			ein Epimorphismus sei. Weiter sei $F\colon\cat{A}\to\ccat{Ab}$ ein
			Funktor und $\alpha_{D_0}\colon \cat{A}(\Gamma D_0, {-})\Rightarrow F$
			eine natürliche Transformation. Dann besitzt $\alpha_{D_0}$ die
			Faktorisierung
			\[
				\begin{tikzcd}
					\cat{A}(\Gamma D_0, {-}) \arrow{r}{s_{D_0}} & \colim_{D\leq D_0}\cat{A}(\Gamma D, {-}) \arrow{r}{\alpha} & F.
				\end{tikzcd}
			\]
			$s_{D_0}$ ist hierbei der kanonische Morphismus des Colimes.
		\end{lemma}
		\begin{theorem}[von der volltreuen Einbettung; Einbettungssatz von Freyd-Mitchell]
			Für jede kleine abelsche Kategorie existieren ein Ring $R$ und eine volltreue
			und exakte Einbettung in die Kategorie $\ccat{Mod}_R$ der $R$-Moduln.
		\end{theorem}
\end{document}
