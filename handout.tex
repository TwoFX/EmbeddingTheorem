\documentclass[a4paper, parskip=half]{scrartcl}

\usepackage[l2tabu, orthodox]{nag}

\usepackage[utf8]{inputenc}
\usepackage[T1]{fontenc}

\usepackage[ngerman]{babel}

\usepackage{amsmath}
\usepackage{amssymb}
\usepackage[thmmarks, amsmath]{ntheorem}
\usepackage{mathtools}
\usepackage{physics}
\usepackage{centernot}

\usepackage{csquotes}
\usepackage{lmodern}
\usepackage{microtype}
\usepackage{enumitem}
\usepackage{stmaryrd}
\usepackage{tikz-cd}
\usepackage{faktor}
\usepackage{etoolbox}
\usepackage{wrapfig}

\usepackage[hidelinks]{hyperref}

\theoremstyle{marginbreak}
\theorembodyfont{\normalfont}
\newtheorem{remark}{Bemerkung}[section]
\newtheorem{proposition}[remark]{Proposition}
\newtheorem{theorem}[remark]{Satz}
\newtheorem{lemma}[remark]{Hilfssatz}
\newtheorem{definition}[remark]{Definition}
\newtheorem{defre}[remark]{Definition und Bemerkung}
\newtheorem{defle}[remark]{Definition und Hilfssatz}
\newtheorem{re}[remark]{Erinnerung}
\newtheorem{cor}[remark]{Folgerung}
\newtheorem{corde}[remark]{Folgerung und Definition}
\newtheorem{example}[remark]{Beispiel}

\theoremstyle{nonumberplain}
\theoremheaderfont{\normalfont\itshape}
\theorembodyfont{\footnotesize}
\theoremsymbol{\ensuremath{\Box}}
\newtheorem{proof}{Beweis.}


\MakeOuterQuote{"}

\newtoggle{proofs}
\toggletrue{proofs}

\newcommand{\defiff}{\mathrel{\vcentcolon\Longleftrightarrow}}
\newcommand\ccat\mathsf
\newcommand\cat\mathcal
\newcommand{\down}[1]{{\downarrow}#1}
\newcommand{\p}[1]{\iftoggle{proofs}{#1}{}}
\newcommand{\n}{\pgfmatrixnextcell}

\DeclareMathOperator{\colim}{colim}
\DeclareMathOperator{\id}{id}

\DeclarePairedDelimiterX\set[1]\lbrace\rbrace{\def\given{\;\delimsize\vert\;}#1}

\subtitle{Seminar Kategorientheorie}
\title{Der Einbettungssatz}
\author{Markus Himmel}
\date{19. Juli 2018}

\begin{document}
	\maketitle
	\setcounter{section}{-1}

	\section{Aufwärmen}
		\begin{proposition}[Explizite Berechnung kleiner filtrierter Colimiten in $\ccat{Set}$]\label{2-13-3}
			Es sei $\cat{C}$ klein und filtriert und $F\colon \cat{C}\to\ccat{Set}$ ein
			Funktor. Es ist $\colim_\cat{C} F = L$, wobei
			\[
				L=\faktor{\coprod_{C\in\cat{C}}FC}{\approx},\quad s_C\colon FC\to L;\ x\mapsto [x]_\approx.
			\]
			Die Äquivalenzrelation $\approx$ ist gegeben durch
			\[
				(x\in FC) \approx (x'\in FC') \defiff \exists C''\in\cat{C}, f\colon C\to C'', g\colon C'\to C'': Ff(x) = Fg(x').
			\]
		\end{proposition}
		\textcolor{red}{Wie genau funktioniert das in $\ccat{Ab}$? Borceux erwähnt nur,
			dass der Vergissfunktor filtrierte Colimiten reflektiert, aber ich schaffe
			es nicht mal, $L$ überhaupt zu einer abelschen Gruppe zu machen.}
		\p{\begin{proof}
			Zu zeigen sind:
			\begin{enumerate}[noitemsep]
				\item $\approx$ ist eine Äquivalenzrelation.
				%\item $L$ ist eine abelsche Gruppe.
				\item Der gegebene Kegel ist ein Grenzkegel.
			\end{enumerate}
			\begin{enumerate}[label=Zu \arabic*:]
					\begin{minipage}{0.7\textwidth}
					\item Klar: $\approx$ ist reflexiv und symmetrisch. Sei $(x\in FC_1)
					\approx (x' \in FC_2)$, $(x'\in FC_2) \approx (x''\in FC_3)$.
					Es existieren also Morphismen $f, g, h, k$, sodass
					$Ff(x) = Fg(x')$, $Fh(x')=Fk(x'')$. Da $\cat{C}$ filtriert,
					existiert ein Cokegel $(\alpha_i\colon C_i\to C_6)$ über dem
					gegebenen Diagramm. Insbesondere haben wir also
					\begin{align*}
						F\alpha_4 \circ Ff(x) &= F\alpha_4\circ Fg(x') = F\alpha_2(x')\\
							&= F\alpha_5\circ Fh(x') = F\alpha_5\circ Fk(x'').
					\end{align*}
					Dies zeigt $x \approx x''$.
					\end{minipage}
					\begin{minipage}{0.25\textwidth}
						\[\begin{tikzcd}[sep=1em]
							C_1 \ar{dr}{f} \n \n\\
							\n C_4 \ar{dr}{\alpha_4} \n\\
							C_2 \ar{dr}{h} \ar{ur}{g} \ar{rr}{\alpha_2} \n \n C_6\\
							\n C_5 \ar{ur}{\alpha_5} \n\\
							C_3 \ar{ur}{k} \n \n
						\end{tikzcd}\]
					\end{minipage}
				%\item Es sei $(x\in FC_1)\approx (x'\in FC_2)$, $(y\in FC_3)\approx(y'\in FC_4)$.
				\item Es seien $C, C'\in\cat{C}$, $f\colon C\to C'$, $x\in C$.
					Wegen $Ff(x) = F(\id_{C'})(Ff(x))$ gilt $s_C(x) = [x]_\approx
					= [Ff(x)]_\approx = s_{C'}(Ff(x)) \implies s_C = s_{C'}\circ Ff$.
					Also bilden die $s_C$ einen Kegel. Es sei ein weiterer Kegel
					gegeben durch $(t_C\colon FC\to M)_{C\in\cat{C}}$. Definiere
					$t\colon L\to M$ durch $t([x]_\approx)\coloneqq t_C(x)$ für
					$x\in FC$. Dies ist wohldefiniert, denn für $x'\in FC'$ mit
					$[x']_\approx = [x]_\approx$ existieren $f\colon C\to C''$,
					$g\colon C'\to C''$ mit $Ff(x) = Fg(x')$. Da $t$ ein Kegel ist, folgt
					\[
						t_C(x) = t_{C''}\circ Ff(x) = t_{C''}\circ Fg(x') = t_{C'}(x').
					\]
					$t\circ s_C=t_C$ gilt per Konstruktion und $t$ wird durch diese
					Identität eindeutig festgelegt.
			\end{enumerate}
		\end{proof}}
		\begin{proposition}\label{2-5-9}
			Es sei $\cat{C}$ eine Kategorie mit Pullbacks.
			\[
				\begin{tikzcd}[sep=4em]
					A \ar{r}{a} \ar[phantom]{dr}{\text{(I)}} \ar{d}{c} & B \ar[phantom]{dr}{\text{(II)}} \ar{r}{b} \ar{d}{d} & C \ar{d}{e}\\
					D \ar{r}{f} & E \ar{r}{g} & F
				\end{tikzcd}
			\]
			Das obige Diagramm sei kommutativ.
			\begin{enumerate}
				\item Falls (I) und (II) Pullbacks sind, so ist auch das äußere
					Rechteck ein Pullback.
				\item Falls (II) und das äußere Quadrat Pullbacks sind, so ist auch
					(I) ein Pullback.
			\end{enumerate}
		\end{proposition}
		\p{\begin{proof}
			\begin{enumerate}
				\item Es seien $x\colon X\to D$, $y\colon X\to C$ mit $g\circ f\circ x = e\circ y$.
					Pullback (II) liefert ein eindeutiges $z\colon X\to B$, sodass
					$b\circ z = y$, $d\circ z = f\circ x$. Letztere Gleichheit und Pullback
					(I) geben ein eindeutiges $w\colon X\to A$, sodass $a\circ w = z$, $c\circ w = x$.
					Insbesondere gilt $b\circ a\circ w = b\circ z = y$. Es sei $w'\colon X\to A$ ein
					weiterer Morphismus mit $b\circ a\circ w' = y$ und $c\circ w' = x$. Es gilt also
					$b\circ (a\circ w') = y = b\circ (a\circ w)$ und
					$d\circ(a\circ w') = f\circ (c\circ w') = f\circ x = f\circ c\circ w = d\circ (a\circ w)$.
					Pullback (II) liefert $a\circ w'=a\circ w$. Weiter ist $c\circ w'= x = c\circ w$.
					Pullback (I) liefert $w=w'$.
				\item Es sei $(A', c', a')$ der Pullback von $f$ und $d$. Mit (1) und da (II) ein Pullback ist, ist $(A', c', b\circ a')$ ein Pullback von
					$(g\circ f, e)$. Es ist jedoch auch $(A, c, b\circ a)$ ein Pullback von $(g\circ f, e)$.
					Da Limiten eindeutig sind, folgt $A\cong A'$. \textcolor{red}{Reicht das? Borceux
					schreibt den Isomophismus noch hin, aber ich glaube, das hier zeigt schon die
					Behauptung.}
			\end{enumerate}
		\end{proof}}
		\begin{proposition}
			In einer Kategorie $\cat{C}$ sei $f\colon A\to B$ der Coequalizer von
			$x, y\colon X\rightrightarrows A$. Weiter existiere der Pullback
			$\alpha, \beta\colon P\rightrightarrows A$ von $(f, f)$. Dann ist
			$f$ auch der Coequalizer von $(\alpha, \beta)$.
		\end{proposition}
		\p{%
			\begin{minipage}{0.75\textwidth}%
			\begin{proof}
				Es sei $f$ der Coequalizer von $(x, y)$ und $(\alpha, \beta)$ der
				Pullback von $(f, f)$. Wegen $f\circ x = f\circ y$ existiert ein
				eindeutiger Morphismus $z\colon X \to P$ mit $\alpha\circ z = x$,
				$\beta\circ z=y$. Ist nun $g\colon A\to C$ ein Morphismus mit
				$g\circ\alpha = g\circ\beta$, so gilt $g\circ x = g\circ\alpha\circ z
				= g\circ\beta\circ z = g\circ y$. Es ist aber $f$ der Coequalizer von
				$(x, y)$, also existiert ein eindeutiges $h$ mit $g = h\circ f$.
				\textcolor{red}{\LaTeX{} macht komische Dinge mit der vertikalen
				Anordnung der Abbildung\dots}
			\end{proof}
		\end{minipage}%
		\begin{minipage}[t][0.25\textwidth][t]{0.25\textwidth}
			\[
				\begin{tikzcd}[sep=1.4em]
					\n X \ar{dl}[']{z} \ar[shift left]{d}{y} \ar[shift right, ']{d}{x} \n \\
					P \ar[shift left]{r}{\alpha} \ar[shift right, ']{r}{\beta} \n \ar{d}{g} A \ar{r}{f} \n \ar{dl}{h} B \\
					\n C \n
				\end{tikzcd}
			\]
		\end{minipage}}

	\section{Der Einbettungssatz}
		Es sei stets $\cat{A}$ eine kleine abelsche Kategorie.
		\begin{proposition}\label{1-14-1}
			Es sei $F\colon\cat{D}\to\ccat{Ab}^{\cat{A}}$ ein filtriertes Diagramm
			linksexakter additiver Funktoren. Dann ist auch $\colim_{D\in\cat{D}} F(D)$
			linksexakt und additiv.
		\end{proposition}
		\p{\begin{proof}
			Ohne.
		\end{proof}}
		\begin{definition}[Die Einbettung]
			Für eine Kategorie $\cat{D}$ und einen Funktor $\phi\colon\cat{D}\to\cat{A}$
			definiere $U\coloneqq\colim_{D\in\cat{D}} (Y\circ\phi)(D)$, wobei $Y$ die
			kontravariante Yoneda-Einbettung bezeichnet.
		\end{definition}
		\begin{lemma}
			Ist $\cat{D}$ kofiltriert, so ist $U$ linksexakt.
		\end{lemma}
		\p{\begin{proof}
			Jeder darstellbare Funktor $\cat{A}(A, {-})\colon\cat{A}\to\ccat{Ab}$
			präserviert Limiten, denn der darstellbare Funktor
			$\cat{A}(A, {-})\colon\cat{A}\to\ccat{Set}$ tut das (Vortrag),
			und der Vergissfunktor $\ccat{Ab}\to\ccat{Set}$ reflektiert Limiten
			(Ohne Beweis? Liegt daran dass er Isomorphismen reflektiert). Also
			ist $\cat{A}(A, {-})\colon \cat{A}\to\ccat{Ab}$ linksexakt (Vortrag).
			Da $Y$ kontravariant ist, ist $U$ ein filtrierter Colimes linksexakter
			Funktoren und mit \ref{1-14-1} selbst linksexakt.
		\end{proof}}
		\begin{lemma}
			Es sei $\cat{D}$ kofiltriert. Weiter besitze jeder Epimorphismus
			$f\colon A\to\phi(D)$ in $\cat{A}$ die Darstellung $f = \phi(d)$, wobei
			$d\colon D'\to D$ in $\cat{D}$. Dann erhält $U$ Epimorphismen.
		\end{lemma}
		\p{\begin{proof}
			Es sei $g\colon B\to C$ ein Epimorphismus in $\cat{A}$. $U(g)$ ist genau
			dann ein Epimorphismus, wenn er surjektiv ist. Wir verwenden \ref{2-13-3}
			(in $\ccat{Ab}$).

			Es sei $x\in U(C) = \colim_{D\in\cat{D}}\cat{A}(\phi(D), C)$. $x$ ist
			die Äquivalenzklasse eines Morphismus $\gamma\colon \phi(D)\to C$ in
			$\cat{A}$. Damit $U(g)$, surjektiv ist, muss $y\in U(B)$ existieren,
			sodass $U(g)(y)=x$. $y$ ist die Äquivalenzklasse eines Morphismus
			$\beta\colon\phi(D')\to B$. Es gilt $U(g)([\beta]_\approx) = [g\circ\beta]_\approx$,
			wir suchen also (Kontravarianz!) $D''\in\cat{D}$, $d\colon D''\to D$,
			$d'\colon D''\to D'$, sodass $g\circ\beta\circ\phi(d')=\gamma\circ\phi(d)$.

			\begin{minipage}{0.5\textwidth}
				\[
					\begin{tikzcd}[sep=1.6em]
						\phi(D') \ar{d}{\beta} \n \phi(D'') \ar[']{l}{\phi(d')} \ar{r}{\phi(d)} \n \phi(D)\ar{d}{\gamma}\\
						B \ar{rr}{g} \n \n C
					\end{tikzcd}
				\]
			\end{minipage}%
			\begin{minipage}{0.5\textwidth}
				\[
					\begin{tikzcd}[sep=1.6em]
						A \ar{d}{\alpha} \ar[two heads]{r}{f} \n \phi(D) \ar{d}{\gamma}\\
						B \ar[two heads]{r}{g} \n C
					\end{tikzcd}
				\]
			\end{minipage}

			Es sei $(A, \alpha, f)$ der Pullback von $(\gamma, g)$ in $\cat{A}$,
			also $g\circ\alpha = \gamma\circ f$.
			$f$ ist ein Epimorphismus, da $g$ ein Epimorphismus (Vortrag), und
			per Voraussetzung existiert $\overline{d}\colon \overline{D}\to D$
			mit $f=\phi(\overline{d})$. Die Wahlen $D'\coloneqq D''\coloneqq \overline{D}$,
			$\beta\coloneqq\alpha$, $d'\coloneqq\id_{\overline{D}}$, $d\coloneqq\overline{d}$
			liefern $g\circ\beta\circ\phi(d')=g\circ\alpha\circ\phi(\id_{\overline{D}})
			= g\circ\alpha = \gamma\circ f = \gamma\circ\phi(\overline{d}) = \gamma\circ\phi(d)$
			wie gewünscht.
		\end{proof}}
		\begin{lemma}
			Es sei $\cat{D}$ kofiltriert. Weiter gelte:
			\begin{enumerate}[label=(\arabic*)]
				\item $\forall A\in\cat{A}\ \exists D\in\cat{D}: A=\phi(D)$;
				\item $\phi(d)$ ist ein Epimorphismus in $\cat{A}$ für jeden Morphismus $d$ in $\cat{D}$.
			\end{enumerate}
			Dann ist $U$ treu.
		\end{lemma}
		\p{\begin{proof}
			Es seien $f\colon A\rightrightarrows B$ Morphismen in $\cat{A}$ mit
			$U(f)=U(g)$. $a\in U(A)$ sei die Äquivalenzklasse von $\alpha\colon\phi(D)\to A$.
			Es ist$[f\circ\alpha]_\approx=U(f)(a)=U(g)(a)=[g\circ\alpha]_\approx$, also
			existiert $d\colon D'\to D$ in $\cat{D}$ mit $f\circ\alpha\circ\phi(d)=g\circ\alpha\circ\phi(d)$.

			Ist nun $\phi(\overline{D})=A$ für ein $\overline{D}\in\cat{D}$ (Voraussetzung),
			so wähle $\alpha\coloneqq\id_{A}$. Wir erhalten $f\circ \id_A\circ\phi(d)=g\circ\id_A\circ\phi(d)$.
			Nach Voraussetzung ist $\phi(d)$ ein Epimorphismus und es folgt $f=g$.
		\end{proof}}
		\begin{theorem}[von der treuen Einbettung]\label{abeb}
			Jede kleine abelsche Kategorie lässt sich treu und exakt in die Kategorie
			der abelschen Gruppen einbetten.
		\end{theorem}
		\p{\begin{proof}
			Wir konstruieren eine kleine Kategorie $\cat{D}$ und einen Funktor
			$\phi\colon\cat{D}\to\cat{A}$, die die Voraussetzungen der vorangegangenen
			Hilfssätze erfüllen. $\cat{D}$ wird eine Poset-Kategorie sein; im Folgenden
			bezeichnen $D$, $D_n$ und $D_n^\alpha$ je nach Kontext entweder die Halbordnung
			selbst oder die zugehörige Kategorie. Ebenso bezeichne $D_1\leq D_2$ je nach Kontext
			die Aussage oder den zugehörigen Morphismus.

			Wir konstruieren eine aufsteigende Folge $\cat{D}_0 \subseteq \cat{D}_1 \subseteq \cdots
			\subseteq \cat{D}_n \subseteq\cdots$ ($n\in\mathbb{N}_0$) von Halbordnungen und
			Funktoren $\phi_n\colon\cat{D}_n\to\cat{A}$, die die folgenden Eigenschaften
			erfüllen:
			\begin{enumerate}[label=(\arabic*),noitemsep]
				\item Für $a, b\in\cat{D}_n$ existiert das Infimum $a\wedge b$;
				\item Für $n\leq m$ fallen $\phi_n$ und $\phi_m$ auf $\cat{D}_n$ zusammen;
				\item $\phi_n(d)$ ist ein Epimorphismus für jedes $n\in\mathbb{N}$ und
					für jeden Morphismus $d$ in $\cat{D}_n$;
				\item $\forall A\in\cat{A}\ \exists D\in\cat{D}_1:\phi_1(D)= A$;
				\item $\forall A\in\cat{A}, D\in\cat{D}_n, f\colon A\twoheadrightarrow\phi_n(D)
					\ \exists d\in\cat{D}_{n+1}, d\colon D'\to D: \phi_{n+1}(d)=f$.
			\end{enumerate}
			Mit $\cat{D}\coloneqq\bigcup_{n\in\mathbb{N}_0}\cat{D}_n$ und der naheliegenden
			Fortsetzung der $\phi_n$ zu $\phi$ sind dann alle Voraussetzungen erfüllt.

			Definiere $\cat{D}_0\coloneqq\set{*}$, $\phi_0(*)\coloneqq0_\cat{A}$.
			Diese Wahlen besitzen offensichtlich die Eigenschaften (1) und (3).

			Sind $\cat{D}_0,\ldots,\cat{D}_n$ bereits definiert und (1), (2), (3)
			und (5) erfüllt, so bezeichnen wir die Paare $(D, f)$, wobei $D\in\cat{D}_n$
			und $f\colon A\twoheadrightarrow \phi_n(D)$, mit den aufeinanderfolgenden
			Successor-Ordinalen. Wir konstruieren nun eine aufsteigende Folge von
			Halbordnungen $\cat{D}_n^0\subseteq\cat{D}_n^1\subseteq\cdots\subseteq\cat{D}_n^\alpha\subseteq\cdots$,
			und eine Folge von Funktoren $\phi_n^\alpha\colon\cat{D}_n^\alpha\to\cat{A}$,
			wobei $\alpha$ bis zum Supremum der für die Paare verwendeten Indizes läuft und
			folgende Eigenschaften erfüllt sein sollen:
			\begin{enumerate}[label=(\alph*),noitemsep]
				\item Für $a, b\in\cat{D}_n^\alpha$ existiert das Infimum $a\wedge b$;
				\item Für $\beta<\alpha$ fallen $\phi_n^\alpha$ und $\phi_n^\beta$ auf $\cat{D}_n^\beta$;
				\item $\phi_n^\alpha(d)$ ist ein Epimorphismus für jedes $\alpha$ und jeden
					Morphismus $d$ in $\cat{D}_n^\alpha$;
				\item Es sei $(D, f)$ das von $\alpha$ indizierte Paar, wobei
					$f\colon A\to\phi_n(D)$. Dann existiert $d\colon D'\to D$ in
					$\cat{D}_n^\alpha$, sodass $\phi_n^\alpha(d) = f$.
			\end{enumerate}
			Mit $\cat{D}_{n+1}\coloneqq\bigcup_\alpha D_n^\alpha$ und der naheliegenden
			Fortsetzung $\phi_{n+1}\colon\cat{D}_{n+1}\to\cat{A}$ sind dann (1), (2), (3)
			und (5) erfüllt.

			$(\cat{D}_n^0, \phi_n^0)\coloneqq (\cat{D}_n, \phi_n)$. Ist $\alpha$ ein
			Limit-Ordinal, wähle $\cat{D}_n^\alpha\coloneqq\bigcup_{\beta<\alpha}\cat{D}_n^\beta$
			und für $\phi_n^\alpha$ die entsprechende Fortsetzung. Es sei $(\cat{D}_n^\alpha,
			\phi_n^\alpha)$ definiert und $\alpha+1$ bezeichne das Paar $(D, f)$.
			Wir betrachten für die Menge $\down{D}\coloneqq\set{D'\in\cat{D}_n^\alpha\given D'\leq D}$
			die disjunkte Vereinigung $\cat{D}_n^\alpha\amalg\down{D}$. $D'^*$
			bezeichne die Kopie von $D'\leq D$ aus $\down{D}$. Wir versehen
			$\cat{D}_n^\alpha\amalg\down{D}$ mit eine Halbordnung, indem wir die
			Ordnungen auf $\cat{D}_n^\alpha$ und $\down{D}$ übernhemen und zusätzlich
			$D'*\leq D'$ für alle $D'\leq D$ fordern. $\cat{D}_n^{\alpha+1}\coloneqq
			\cat{D}_n^\alpha\amalg\down{D}$ mit der eben beschriebenen Halbordnung.
			In $\down{D}$ existieren Infima. Für $D_1\in\down{D}, D_2\in\cat{D}_n^\alpha$
			haben wir $(D_1\wedge D_2)^*=D_1^*\wedge D_2$ in $\cat{D}_n^{\alpha+1}$.

			Auf $\cat{D}_n^\alpha$ definieren wir $\phi_{n+1}^\alpha$ als gleich
			$\phi_n^\alpha$. Weiter definieren wir $\phi_n^{\alpha+1}(D^*)\coloneqq A$
			und $\phi_n^{\alpha+1}(D^*\leq D)\coloneqq f$.
			\[
				\begin{tikzcd}[column sep=8em]
					P \eqqcolon \phi_n^{\alpha+1}(D'^*)
						\ar[two heads]{d}{v\eqqcolon \phi_n^{\alpha + 1}(D'^*\leq D')}
						\ar[two heads]{r}{u\eqqcolon \phi_n^{\alpha+1}(D'^*\leq D^*)}
					\n A=\phi_n^{\alpha+1}(D^*)
						\ar[two heads]{d}{f = \phi_n^{\alpha+1}(D^*\leq D)}\\
					\phi_n^\alpha(D') \ar[two heads]{r}{\phi_n^\alpha(D'\leq D)} \n \phi_n(D)
				\end{tikzcd}
			\]

			Für $D'\leq D$ definieren wir $(\phi_n^{\alpha+1}, \phi_n^{\alpha+1}(D'^*\leq D'),
			\phi_n^{\alpha+1}(D'^*\leq D^*))$ durch den Pullback
			von $(f, \phi_n^\alpha(D'\leq D))$ in $\cat{A}$. Die durch den Pullback
			definierten Morphismen sind Epimorphismen, weil $f$ und $\phi_n^\alpha(D'\leq D)$
			Epimorphismen sind.
			\[
				\begin{tikzcd}[column sep=9em]
					P=\phi_n^{\alpha+1}(D'^*)
						\ar[two heads]{d}{x=\phi_n^{\alpha+1}(D'^*\leq D')}
						\ar[two heads]{r}{y\eqqcolon \phi_n^{\alpha+1}(D'^*\leq D''^*)} \n
					\phi_n^{\alpha+1}(D''^*)
						\ar[two heads]{d}{}
						\ar[two heads]{r}{} \n
					A
						\ar[two heads]{d}{f} \\
					\phi_n^\alpha(D')
						\ar[two heads]{r}{\phi_n^\alpha(D'\leq D'')} \n
					\phi_n^\alpha(D'')
						\ar[two heads]{r}{\phi_n^\alpha(D''\leq D)} \n
					\phi_n(D)
				\end{tikzcd}
			\]
			Für $D'\leq D''\leq D$ in $\cat{D}_n^\alpha$ betrachte obiges Diagramm,
			in dem beide Quadrate Pullbacks sind. Das äußere Quadrat ist ein Pullback
			und es ist $\phi_n^\alpha(D''\leq D)\circ\phi_n^\alpha(D'\leq D'')=\phi_n^\alpha(D'\leq D)$,
			daher folgt $P=\phi_n^{\alpha+1}(D'^*)$
			\textcolor{red}{(wieso steht hier \enquote{$=$} und nicht \enquote{$\cong$}? Pullbacks sind doch nur
			bis auf Isomorphie eindeutig.)}
			und $x=\phi_n^{\alpha+1}(D'^*\leq D')$
			nach vorangegangener Definiton. Wir definieren $\phi_n^{\alpha+1}(D'^*\leq D''^*)\coloneqq y$.

			Für $D'\leq D$, $D'\leq D''$ in $\cat{D}_n^\alpha$ definiere
			$\phi_n^{\alpha+1}(D'^*\leq D'')\coloneqq \phi_n^{\alpha+1}(D'\leq D'')
			\circ\phi_n^{\alpha+1}(D'^*\leq D')$.

			$\phi_n^{\alpha+1}$ ist ein Funktor aufgrund der Verknüpfungseigenschaften
			von Pullbacks \textcolor{red}{(siehe oben)}.
			Die Konstruktion der $\cat{D}_n^\alpha$, $\phi_n^\alpha$ stellt sicher,
			dass die Eigenschaften gegeben sind. Insbesondere haben wir
			$\phi_n^{\alpha+1}(D^*\leq D)=f$.

			Wir zeigen noch Eigenschaft (4). Die Konstruktion von $\cat{D}_1$ läuft
			über alle Paare $(*, f)$, wobei $f\colon A\to 0_\cat{A}$ ein Epimorphismus.
			Es ist aber jeder Morphismus $A\to 0_\cat{A}$ ein Epimorphismus, denn
			$0_\cat{A}$ ist initial, das heißt wenn $u\circ f=v\circ f$, dann kommt für $u$ und $v$
			überhaupt nur ein Morphismus in Frage. Da $0_\cat{A}$ terminal ist,
			haben wir für jedes $A$ in $\cat{A}$ ein entsprechendes Paar $(*, f)$,
			und Forderung (d) liefert, was wir brauchen.
		\end{proof}}

		Die Aussagen von \ref{fin}, \ref{fact} und \ref{func} beziehen sich auf
		die im Beweis von \ref{abeb} konstruierten
		Wahlen für $\cat{D}$ und $\phi$. \textcolor{red}{Ergibt es überhaupt Sinn,
		diese Sätze auf dem Handout zu haben, wenn der Beweis von \ref{abeb}
		nicht drauf ist?}
		\begin{defle}\label{fin}
			Für alle Objekte $D_1,D_2\in\cat{D}$ ist das Segment
				\[
					[D_1, D_2] \coloneqq\set{D\in\cat{D}\given D_1\leq D\leq D_2}
				\]
			endlich.
		\end{defle}
		\p{\begin{proof}
			Wir spielen die beiden verschachtelten Induktionen noch einmal durch und
			zeigen für zwei Elemente $D_1, D_2$, die an einer Stelle der Konstruktion
			zusammen auftauchen:
			\begin{enumerate}[label=(\arabic*),noitemsep]
				\item $[D_1, D_2]$ ist endlich,
				\item $[D_1, D_2]$ wird sich in keinem weiteren Schritt noch
					verändern.
			\end{enumerate}
			$\cat{D}_0 = \set{*}$ erfüllt (1). Falls $\cat{D}_n^\alpha$ die Bedingungen
			erfüllt, betrachte $D_1, D_2$ in $\cat{D}_n^{\alpha+1}$. Die Konstruktion
			von $\cat{D}_n^{\alpha+1}$ stellt sicher, dass $D'\leq D^*$ für ein Element
			$D'$ in $\cat{D}_n^\alpha$ und ein Element $D^*$, das in $\cat{D}_n^\alpha$
			neu dazukommt, nicht vorkommen kann.

			Es seien $D_1, D_2\in\cat{D}_n^{\alpha+1}$, $D_1\leq D_2$.
			Falls $D_1, D_2\in\cat{D}_n^\alpha$,
			so ist $[D_1, D_2]$ in $\cat{D}_n^{\alpha+1}$ unverändert gegenüber
			$\cat{D}_n^{\alpha}$. Falls $D_1, D_2\notin\cat{D}_n^{\alpha}$,
			dann existieren $D, D'\in\cat{D}_n^\alpha$ mit $D_1=D^*, D_2=D'^*$.
			$[D_1,D_2]$ in $\cat{D}_n^{\alpha+1}$ ist dann isomorph zu $[D, D']$
			in $\cat{D}_n^\alpha$, also nach Voraussetzung endlich.
			Falls $D_1\notin\cat{D}_n^\alpha$, $D_2\in\cat{D}_n^\alpha$, dann ist
			$D_1=D^*$ und $[D_1, D_2] = [D^*, D_2^*]\cup[D, D_2]$, also endlich.
			$D_1\in\cat{D}_n^\alpha$, $D_2\notin\cat{D}_n^\alpha$ kann nicht eintreten.

			Da die Bedingungen bei jedem Successor-Ordinal erhalten bleiben, bleiben
			sie auch bei einem Limit-Ordinal erhalten (der Limit-Schritt fügt ja gar
			keine Objekte hinzu).
		\end{proof}}
		\begin{lemma}\label{fact}
			Sind $D_1, D_2\in\cat{D}$ Objekte mit $D_1\leq D_2$, so ist die kanonische Faktorisierung
			\[
				\begin{tikzcd}
					\phi(D_1) \arrow[r] & \lim_{D_1<D\leq D_2}\phi(D)
				\end{tikzcd}
			\]
			ein Epimorphismus.
		\end{lemma}
		\p{\begin{proof}
			Wieder mit Induktion über den Moment, in dem $D_1$ eingeführt wird. Da in
			einem späteren Schritt keine Elemente $D'$ mit $D_1\leq D'$ eingeführt werden,
			sind in diesem Schritt bereits alle Elemente der Form $D_1<D\leq D_2$ vorhanden.

			Falls $D_1\in\cat{D}_0$, so folgt $D_1=*=D_2$, und der Limes über die
			leere Kategorie ist gerade $0_\cat{A}$. Der Morphismus $\phi(D_1)\to 0_\cat{A}$ ist
			ein Epimorphismus wie eben beobachtet.

			Bei einem Limit-Schritt kommen keine neuen Objekte hinzu, also verbleibt der Fall,
			dass $D_1$ in der Konstruktion von $\cat{D}_n^{\alpha+1}$ eingeführt wird. Es existiert
			ein $D_0\in\cat{D}_n^\alpha$ mit $D_1=D_0^*$. Es ist nun entweder $D_2\in\cat{D}_n^\alpha$
			oder $D_2\notin\cat{D}_n^\alpha$.

			Falls $D_1=D_2\notin\cat{D}_n^\alpha$ sind wir in der gleichen Situation wie im Fall
			$D_1\in\cat{D}_0$.
			\[
				\begin{tikzcd}[column sep=5em, row sep=3em]
					\phi(D_0^*)
						\ar{d}{}
						\ar[phantom]{dr}{\text{(1)}}
						\ar[two heads]{r}{f^*} \n
					\lim_{D_0^*<D^*\leq D_{00}^*}\phi(D^*)
						\ar[phantom]{dr}{\text{(2)}}
						\ar{d}{}
						\ar{r}{p_{D^*}} \n
					\phi(D^*)
						\ar[phantom]{dr}{\text{(3)}}
						\ar{d}{}
						\ar{r}{\phi(D^*\leq D_{00}^*)} \n
					\phi(D_{00}^*)
						\ar{d}{} \\
					\phi(D_0)
						\ar[two heads]{r}{f} \n
					\lim_{D_0<D\leq D_{00}}\phi(D)
						\ar{r}{p_D} \n
					\phi(D)
						\ar{r}{\phi(D\leq D_{00})} \n
					\phi(D_{00})
				\end{tikzcd}
			\]
			Falls $D_1\neq D_2\notin \cat{D}_n^\alpha$, $D_2=D_{00}^*$ für ein
			$D_{00}\in\cat{D}_n^\alpha$, betrachte das Diagramm, wobei
			$D$ beliebig mit $D_1<D\leq D_2$. (3) ist ein Pullback
			nach Konstruktion. Da ein Grenzkegel ein Kegel ist, gilt
			$\phi(D\leq D_{00})\circ p_D = p_{D_{00}}$, $\phi(D^*\leq D_{00}^*)\circ p_{D^*}
			= p_{D_{00}^*}$. Wir haben also ein Rechteck (2-3), bei dem
			es sich gerade um den Limes der Pullbacks (3) handelt \textcolor{red}{(warum?)}. Durch Vertauschung
			von Limiten ist auch (2-3) ein Pullback. Da aber auch das äußere Rechteck
			nach Konstruktion ein Pullback ist, liefert \ref{2-5-9}, dass auch (1) ein
			Pullback ist. Nach Induktionsvoraussetzung ist $f$ ein Epimorphismus, also ist auch
			$f^*$ ein Epimorphismus.

			Der verbleibende Fall ist $D_1=D_0^*$, $D_2\in\cat{D}_n^\alpha$. Der Index
			$\alpha + 1$ bezeichne das Paar $(\overline{D}, \overline{f})$. Falls
			$D_0 = \overline{D}$, so ist $D_1$ das einzige Objekt aus $[D_1, D_2]\cap\down{\overline{D}}$,
			also gilt $\set{D\given D_1\leq D\leq D_2}=[\overline{D}, D_2]$. Dieses Segment
			hat aber das initiale Objekt $\overline{D}$, also ist der gesuchte Limes gerade
			$\phi(\overline{D})$ (ohne Beweis: Hat $\cat{C}$ ein initiales Objekt $0_\cat{C}$, so ist
			$\lim_\cat{C} F = F(0_\cat{C})$) und die Faktorisierung ist $\overline{f}$ nach
			Konstruktion.

			Falls $D_1=D_0^*<\overline{D}^*$ und $D_2\in\cat{D}_n^\alpha$, so ziegen wir,
			dass der Limes gerade $\phi(D_1)$ ist und die Faktorisierung durch die Identität gegeben
			ist. Sei dazu $D_3\coloneqq D_2\wedge\overline{D}\in\cat{D}_n^\alpha$.
		\end{proof}}
		\begin{lemma}\label{func}
			Es sei $D_0\in\cat{D}$ ein Objekt und $\Gamma\colon \down{D_0}\to\cat{A}$
			ein Funktor, wobei für alle $D_1\leq D_2\leq D_0$ die Faktorisierung
			\[
				\begin{tikzcd}
					\Gamma(D_1) \arrow[r] & \lim_{D_1<D\leq D_2} \Gamma(D)
				\end{tikzcd}
			\]
			ein Epimorphismus sei. Weiter sei $F\colon\cat{A}\to\ccat{Ab}$ ein
			Funktor und $\alpha_{D_0}\colon \cat{A}(\Gamma D_0, {-})\Rightarrow F$
			eine natürliche Transformation. Dann besitzt $\alpha_{D_0}$ die
			Faktorisierung
			\[
				\begin{tikzcd}
					\cat{A}(\Gamma D_0, {-}) \arrow{r}{s_{D_0}} & \colim_{D\leq D_0}\cat{A}(\Gamma D, {-}) \arrow{r}{\alpha} & F.
				\end{tikzcd}
			\]
			$s_{D_0}$ ist hierbei der kanonische Morphismus des Colimes.
		\end{lemma}
		\begin{theorem}[von der volltreuen Einbettung; Einbettungssatz von Freyd-Mitchell]
			Für jede kleine abelsche Kategorie existieren ein Ring $R$ und eine volltreue
			und exakte Einbettung in die Kategorie $\ccat{Mod}_R$ der $R$-Moduln.
		\end{theorem}
\end{document}
